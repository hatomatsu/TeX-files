% \documentclass[a5paper,8pt,landscape,twocolumn]{ltjsarticle}
% \usepackage[utf8]{inputenc}

%--------------------------------------------------------------------------------------
%margin--------------------------------------------------------------------------------
%--------------------------------------------------------------------------------------
\usepackage[margin=20truemm]{geometry} 


%--------------------------------------------------------------------------------------
%packages------------------------------------------------------------------------------
%--------------------------------------------------------------------------------------
\usepackage{amsmath, amssymb, amsfonts, latexsym, mathtools}
\usepackage[amsmath,amsthm,thmmarks]{ntheorem}
\usepackage{bm}
\usepackage[dvipdfmx]{graphicx}
\usepackage{physics}
\usepackage{xcolor}
\usepackage{graphicx}
\usepackage{thmtools}
\usepackage{cases}
\usepackage{arydshln} % dotted line in matrix
\setlength{\dashlinegap}{2pt}
\setlength{\dashlinedash}{2pt}
\usepackage{multicol,lipsum} % multiple columns
\usepackage{hyperref} %reference
\usepackage{mdframed} %for theorem frame

\usepackage{tikz}
\usetikzlibrary{angles,quotes,shapes.geometric} % for pic
\usetikzlibrary{positioning, calc}
\usepackage[outline]{contour} % glow around text
\contourlength{1.2pt}

\usepackage[inline]{enumitem}
\usepackage{blindtext}
\usepackage{multicol}
\usepackage{etoolbox}
\makeatletter
\patchcmd{\hdots@for}{\hfill}{\hskip\z@\@plus 1filll}{}{}
\makeatother



\theorembodyfont{\normalfont} % This prevents italic font in theorem
%--------------------------------------------------------------------------------------
%  Normal enviroment-----------------------------------------------------------------------------
%--------------------------------------------------------------------------------------
\newtheorem{theorem}{Theorem}[section]
\newtheorem{definition}[theorem]{定義}
\newtheorem{corollary}[theorem]{系}
\newtheorem{lemma}[theorem]{補題}
\newtheorem{proposition}[theorem]{命題}
\newtheorem{example}[theorem]{例}

%--------------------------------------------------------------------------------------
% Framed Enviroment-----------------------------------------------------------------------------
%--------------------------------------------------------------------------------------
\newmdtheoremenv[linewidth=1pt]{mydef}[theorem]{定義}
\newmdtheoremenv[linewidth=1pt]{mythm}[theorem]{定理}
\newmdtheoremenv[linewidth=1pt]{myprop}[theorem]{命題}
\newmdtheoremenv[linewidth=1pt]{mylem}[theorem]{補題}
\newmdtheoremenv[linewidth=1pt]{mycol}[theorem]{系}


%--------------------------------------------------------------------------------------
%Math Symbols--------------------------------------------------------------------------
%--------------------------------------------------------------------------------------
\newcommand{\R}{\mathbb R}
\newcommand{\N}{\mathbb N}
\newcommand{\C}{\mathbb C}
\newcommand{\Z}{\mathbb Z}
\newcommand{\D}{\mathbb D}
\newcommand{\E}{\mathbb E}
\newcommand{\di}{\text{D}}
\newcommand{\A}{\mathbb A}
\newcommand{\SO}{SO}
\newcommand{\M}{\mathsf{M}}
\newcommand{\F}{\mathbb F}
\newcommand{\calF}{\mathcal{F}}
\newcommand{\calS}{\mathcal{S}}
\newcommand{\calH}{\mathcal{H}}
\newcommand{\calB}{\mathcal{B}}
\newcommand{\calC}{\mathcal{C}}
\newcommand{\calD}{\mathcal{D}}
\newcommand{\End}{\text{End}}
\newcommand{\gee}{\mathfrak{g}}
\newcommand{\haa}{\mathfrak{h}}
\newcommand{\kaa}{\mathfrak{k}}
\newcommand{\jee}{\mathfrak{J}}
\newcommand{\slnc}{\mathfrak{sl}_n(\mathbb{C})}
\newcommand{\glnc}{\mathfrak{gl}_n(\mathbb{C})}
\newcommand{\dprime}{{\prime\prime}}
\newcommand{\thmtext}{Text of theorem.}
\newcommand{\class}{\mathcal{C}}
\newcommand{\gl}{GL} %general linear group
\newcommand{\spl}{SL}%Special linear group
% \newcommand{\hom}{Hom}

%--------------------------------------------------------------------------------------
%Math Operators------------------------------------------------------------------------
%--------------------------------------------------------------------------------------
\DeclareMathOperator{\sgn}{sgn}%signature 
\DeclareMathOperator{\Ima}{Im} %image
\DeclareMathOperator{\id}{id} %identity map
\DeclareMathOperator{\proj}{proj} %projection
\DeclareMathOperator{\ad}{ad}
\DeclareMathOperator{\spn}{span} 
\DeclareMathOperator{\rk}{rk} 
\newcommand*{\tran}{^{\mkern-1mu t}} %transpose #1
\newcommand{\transpose}{^{\raisebox{.2ex}{$\intercal$}}}% transpose #2
\newcommand{\eqdef}{\overset{\mathrm{def}}{=\joinrel=}}% def equal
\newcommand{\conj}[1]{%
  \overline{#1}%
} %conjugation
\newcommand{\inv}{^{\raisebox{.2ex}{\large{$\scriptscriptstyle{-1}$}}}}% inverse
% \newcommand{\dprime}{^{\prime\prime}}


\def\vdotfill#1{\vtop to0pt{\null \dimen0=#1\baselineskip\advance\dimen0 by-.4ex 
\kern-1.6ex \cleaders\hbox{\lower.4ex\vbox to1ex{}.}\vskip\dimen0 \vss}}

%--------------------------------------------------------------------------------------
% Tikz------------------------------------------------------------------------
%--------------------------------------------------------------------------------------

\tikzset{>=latex} % for LaTeX arrow head
\tikzstyle{vector}=[->,very thick] %For vector

